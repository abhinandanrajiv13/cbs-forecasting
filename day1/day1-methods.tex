\documentclass[ignorenonframetext,]{beamer}
\setbeamertemplate{caption}[numbered]
\setbeamertemplate{caption label separator}{: }
\setbeamercolor{caption name}{fg=normal text.fg}
\beamertemplatenavigationsymbolsempty
\usepackage{lmodern}
\usepackage{amssymb,amsmath}
\usepackage{ifxetex,ifluatex}
\usepackage{fixltx2e} % provides \textsubscript
\ifnum 0\ifxetex 1\fi\ifluatex 1\fi=0 % if pdftex
  \usepackage[T1]{fontenc}
  \usepackage[utf8]{inputenc}
\else % if luatex or xelatex
  \ifxetex
    \usepackage{mathspec}
  \else
    \usepackage{fontspec}
  \fi
  \defaultfontfeatures{Ligatures=TeX,Scale=MatchLowercase}
\fi
% use upquote if available, for straight quotes in verbatim environments
\IfFileExists{upquote.sty}{\usepackage{upquote}}{}
% use microtype if available
\IfFileExists{microtype.sty}{%
\usepackage{microtype}
\UseMicrotypeSet[protrusion]{basicmath} % disable protrusion for tt fonts
}{}
\newif\ifbibliography
\hypersetup{
            pdftitle={Day 1 - Methods},
            pdfauthor={Klaus Langenheldt},
            pdfborder={0 0 0},
            breaklinks=true}
\urlstyle{same}  % don't use monospace font for urls
\usepackage{color}
\usepackage{fancyvrb}
\newcommand{\VerbBar}{|}
\newcommand{\VERB}{\Verb[commandchars=\\\{\}]}
\DefineVerbatimEnvironment{Highlighting}{Verbatim}{commandchars=\\\{\}}
% Add ',fontsize=\small' for more characters per line
\usepackage{framed}
\definecolor{shadecolor}{RGB}{248,248,248}
\newenvironment{Shaded}{\begin{snugshade}}{\end{snugshade}}
\newcommand{\KeywordTok}[1]{\textcolor[rgb]{0.13,0.29,0.53}{\textbf{#1}}}
\newcommand{\DataTypeTok}[1]{\textcolor[rgb]{0.13,0.29,0.53}{#1}}
\newcommand{\DecValTok}[1]{\textcolor[rgb]{0.00,0.00,0.81}{#1}}
\newcommand{\BaseNTok}[1]{\textcolor[rgb]{0.00,0.00,0.81}{#1}}
\newcommand{\FloatTok}[1]{\textcolor[rgb]{0.00,0.00,0.81}{#1}}
\newcommand{\ConstantTok}[1]{\textcolor[rgb]{0.00,0.00,0.00}{#1}}
\newcommand{\CharTok}[1]{\textcolor[rgb]{0.31,0.60,0.02}{#1}}
\newcommand{\SpecialCharTok}[1]{\textcolor[rgb]{0.00,0.00,0.00}{#1}}
\newcommand{\StringTok}[1]{\textcolor[rgb]{0.31,0.60,0.02}{#1}}
\newcommand{\VerbatimStringTok}[1]{\textcolor[rgb]{0.31,0.60,0.02}{#1}}
\newcommand{\SpecialStringTok}[1]{\textcolor[rgb]{0.31,0.60,0.02}{#1}}
\newcommand{\ImportTok}[1]{#1}
\newcommand{\CommentTok}[1]{\textcolor[rgb]{0.56,0.35,0.01}{\textit{#1}}}
\newcommand{\DocumentationTok}[1]{\textcolor[rgb]{0.56,0.35,0.01}{\textbf{\textit{#1}}}}
\newcommand{\AnnotationTok}[1]{\textcolor[rgb]{0.56,0.35,0.01}{\textbf{\textit{#1}}}}
\newcommand{\CommentVarTok}[1]{\textcolor[rgb]{0.56,0.35,0.01}{\textbf{\textit{#1}}}}
\newcommand{\OtherTok}[1]{\textcolor[rgb]{0.56,0.35,0.01}{#1}}
\newcommand{\FunctionTok}[1]{\textcolor[rgb]{0.00,0.00,0.00}{#1}}
\newcommand{\VariableTok}[1]{\textcolor[rgb]{0.00,0.00,0.00}{#1}}
\newcommand{\ControlFlowTok}[1]{\textcolor[rgb]{0.13,0.29,0.53}{\textbf{#1}}}
\newcommand{\OperatorTok}[1]{\textcolor[rgb]{0.81,0.36,0.00}{\textbf{#1}}}
\newcommand{\BuiltInTok}[1]{#1}
\newcommand{\ExtensionTok}[1]{#1}
\newcommand{\PreprocessorTok}[1]{\textcolor[rgb]{0.56,0.35,0.01}{\textit{#1}}}
\newcommand{\AttributeTok}[1]{\textcolor[rgb]{0.77,0.63,0.00}{#1}}
\newcommand{\RegionMarkerTok}[1]{#1}
\newcommand{\InformationTok}[1]{\textcolor[rgb]{0.56,0.35,0.01}{\textbf{\textit{#1}}}}
\newcommand{\WarningTok}[1]{\textcolor[rgb]{0.56,0.35,0.01}{\textbf{\textit{#1}}}}
\newcommand{\AlertTok}[1]{\textcolor[rgb]{0.94,0.16,0.16}{#1}}
\newcommand{\ErrorTok}[1]{\textcolor[rgb]{0.64,0.00,0.00}{\textbf{#1}}}
\newcommand{\NormalTok}[1]{#1}
\usepackage{graphicx,grffile}
\makeatletter
\def\maxwidth{\ifdim\Gin@nat@width>\linewidth\linewidth\else\Gin@nat@width\fi}
\def\maxheight{\ifdim\Gin@nat@height>\textheight0.8\textheight\else\Gin@nat@height\fi}
\makeatother
% Scale images if necessary, so that they will not overflow the page
% margins by default, and it is still possible to overwrite the defaults
% using explicit options in \includegraphics[width, height, ...]{}
\setkeys{Gin}{width=\maxwidth,height=\maxheight,keepaspectratio}

% Prevent slide breaks in the middle of a paragraph:
\widowpenalties 1 10000
\raggedbottom

\AtBeginPart{
  \let\insertpartnumber\relax
  \let\partname\relax
  \frame{\partpage}
}
\AtBeginSection{
  \ifbibliography
  \else
    \let\insertsectionnumber\relax
    \let\sectionname\relax
    \frame{\sectionpage}
  \fi
}
\AtBeginSubsection{
  \let\insertsubsectionnumber\relax
  \let\subsectionname\relax
  \frame{\subsectionpage}
}

\setlength{\parindent}{0pt}
\setlength{\parskip}{6pt plus 2pt minus 1pt}
\setlength{\emergencystretch}{3em}  % prevent overfull lines
\providecommand{\tightlist}{%
  \setlength{\itemsep}{0pt}\setlength{\parskip}{0pt}}
\setcounter{secnumdepth}{0}

\title{Day 1 - Methods}
\author{Klaus Langenheldt}
\date{03/05/2018}

\begin{document}
\frame{\titlepage}

\begin{frame}[fragile]

\begin{block}{Data Handling \& Visualisation}

\begin{Shaded}
\begin{Highlighting}[]
\CommentTok{# load data}
\NormalTok{data =}\StringTok{ }\NormalTok{AirPassengers}

\CommentTok{# basic statistics}
\KeywordTok{summary}\NormalTok{(data)}
\end{Highlighting}
\end{Shaded}

\begin{verbatim}
##    Min. 1st Qu.  Median    Mean 3rd Qu.    Max. 
##   104.0   180.0   265.5   280.3   360.5   622.0
\end{verbatim}

\end{block}

\begin{block}{Visualisation}

\begin{Shaded}
\begin{Highlighting}[]
\KeywordTok{plot}\NormalTok{(AirPassengers)}
\end{Highlighting}
\end{Shaded}

\includegraphics{day1-methods_files/figure-beamer/AirPassengers-1.pdf}

\end{block}

\begin{block}{Data Inspection}

\begin{Shaded}
\begin{Highlighting}[]
\KeywordTok{plot}\NormalTok{(}\KeywordTok{density}\NormalTok{(AirPassengers))}
\end{Highlighting}
\end{Shaded}

\includegraphics{day1-methods_files/figure-beamer/unnamed-chunk-1-1.pdf}

\end{block}

\begin{block}{Decompositon}

\begin{Shaded}
\begin{Highlighting}[]
\KeywordTok{library}\NormalTok{(forecast)}
\NormalTok{fit <-}\StringTok{ }\KeywordTok{stl}\NormalTok{(AirPassengers, }\DataTypeTok{t.window=}\DecValTok{15}\NormalTok{, }\DataTypeTok{s.window=}\StringTok{"periodic"}\NormalTok{, }\DataTypeTok{robust=}\OtherTok{TRUE}\NormalTok{)}
\KeywordTok{plot}\NormalTok{(fit)}
\end{Highlighting}
\end{Shaded}

\includegraphics{day1-methods_files/figure-beamer/unnamed-chunk-2-1.pdf}

\end{block}

\begin{block}{Seasonal Analysis}

\begin{Shaded}
\begin{Highlighting}[]
\KeywordTok{boxplot}\NormalTok{(AirPassengers}\OperatorTok{~}\KeywordTok{cycle}\NormalTok{(AirPassengers))}
\end{Highlighting}
\end{Shaded}

\includegraphics{day1-methods_files/figure-beamer/unnamed-chunk-3-1.pdf}

\end{block}

\end{frame}

\begin{frame}[fragile]{The \texttt{forecast} package}

\begin{block}{The \texttt{forecast} package}

\begin{itemize}
\tightlist
\item
  The holy grail of R time series forecasting
\item
  Includes all the traditional models
\item
  AR
\item
  MA
\item
  ARIMA
\item
  \ldots{}
\end{itemize}

\end{block}

\begin{block}{Basic Forecasting Methods}

Simple Average \[\hat{y}_{T+h|T} = \bar{y} = (y_{1}+\dots+y_{T})/T\]
Naïve Method \[\hat{y}_{T+h|T} = y_{T}\] Seasonal Naïve Method
\[\hat{y}_{T+h|T} = y_{T+h-m(k+1)}\]

\end{block}

\begin{block}{Comparing Basic Forecasting Methods}

\includegraphics{day1-methods_files/figure-beamer/unnamed-chunk-4-1.pdf}

\end{block}

\end{frame}

\begin{frame}[fragile]{That's TS Forecasting}

\begin{block}{Predict using Moving Average (MA)}

\begin{Shaded}
\begin{Highlighting}[]
\NormalTok{data =}\StringTok{ }\KeywordTok{ts}\NormalTok{(AirPassengers) }\CommentTok{# convert to ts object}
\NormalTok{model =}\StringTok{ }\KeywordTok{ma}\NormalTok{(data, }\DataTypeTok{order=}\DecValTok{2}\NormalTok{)}
\NormalTok{forecast =}\StringTok{ }\KeywordTok{forecast}\NormalTok{(model, }\DataTypeTok{h=}\DecValTok{10}\NormalTok{)}
\KeywordTok{plot}\NormalTok{(forecast)}
\end{Highlighting}
\end{Shaded}

\includegraphics{day1-methods_files/figure-beamer/unnamed-chunk-5-1.pdf}

\end{block}

\begin{block}{Predict using Moving Average (MA)}

\begin{Shaded}
\begin{Highlighting}[]
\NormalTok{data =}\StringTok{ }\KeywordTok{ts}\NormalTok{(AirPassengers) }\CommentTok{# convert to ts object}
\NormalTok{model =}\StringTok{ }\KeywordTok{ma}\NormalTok{(data, }\DataTypeTok{order=}\DecValTok{2}\NormalTok{)}
\NormalTok{forecast =}\StringTok{ }\KeywordTok{forecast}\NormalTok{(model, }\DataTypeTok{h=}\DecValTok{30}\NormalTok{)}
\KeywordTok{plot}\NormalTok{(forecast)}
\end{Highlighting}
\end{Shaded}

\includegraphics{day1-methods_files/figure-beamer/unnamed-chunk-6-1.pdf}

\end{block}

\end{frame}

\begin{frame}[fragile]{That's also TS Forecasting}

\begin{block}{Prediction Intervals}

\includegraphics{day1-methods_files/figure-beamer/unnamed-chunk-7-1.pdf}

\url{https://www.otexts.org/fpp2/prediction-intervals.html}

\end{block}

\begin{block}{Predict using Auto Regression (AR)}

\begin{block}{AR(5)}

\begin{Shaded}
\begin{Highlighting}[]
\NormalTok{data =}\StringTok{ }\KeywordTok{ts}\NormalTok{(AirPassengers) }\CommentTok{# convert to ts object}
\NormalTok{model =}\StringTok{ }\KeywordTok{ar}\NormalTok{(data, }\DataTypeTok{order=}\DecValTok{5}\NormalTok{)}
\NormalTok{forecast =}\StringTok{ }\KeywordTok{forecast}\NormalTok{(model, }\DataTypeTok{h=}\DecValTok{30}\NormalTok{)}
\KeywordTok{plot}\NormalTok{(forecast)}
\end{Highlighting}
\end{Shaded}

\includegraphics{day1-methods_files/figure-beamer/unnamed-chunk-8-1.pdf}

\end{block}

\end{block}

\begin{block}{Predict using Auto Regression (AR)}

\begin{block}{AR(10)}

\begin{Shaded}
\begin{Highlighting}[]
\NormalTok{data =}\StringTok{ }\KeywordTok{ts}\NormalTok{(AirPassengers) }\CommentTok{# convert to ts object}
\NormalTok{model =}\StringTok{ }\KeywordTok{ar}\NormalTok{(data, }\DataTypeTok{order=}\DecValTok{10}\NormalTok{)}
\NormalTok{forecast =}\StringTok{ }\KeywordTok{forecast}\NormalTok{(model, }\DataTypeTok{h=}\DecValTok{30}\NormalTok{)}
\KeywordTok{plot}\NormalTok{(forecast)}
\end{Highlighting}
\end{Shaded}

\includegraphics{day1-methods_files/figure-beamer/unnamed-chunk-9-1.pdf}

\end{block}

\end{block}

\begin{block}{Predict using Auto Regression (AR)}

Higher order, higher pattern recognition (now AR(13))

\begin{Shaded}
\begin{Highlighting}[]
\NormalTok{data =}\StringTok{ }\KeywordTok{ts}\NormalTok{(AirPassengers) }\CommentTok{# convert to ts object}
\NormalTok{model =}\StringTok{ }\KeywordTok{ar}\NormalTok{(data, }\DataTypeTok{order=}\DecValTok{13}\NormalTok{)}
\NormalTok{forecast =}\StringTok{ }\KeywordTok{forecast}\NormalTok{(model, }\DataTypeTok{h=}\DecValTok{30}\NormalTok{)}
\KeywordTok{plot}\NormalTok{(forecast)}
\end{Highlighting}
\end{Shaded}

\includegraphics{day1-methods_files/figure-beamer/unnamed-chunk-10-1.pdf}

\end{block}

\begin{block}{Predict using Auto Regression (AR)}

Long time-frames: mean-convergence and zero trend

\begin{Shaded}
\begin{Highlighting}[]
\NormalTok{data =}\StringTok{ }\KeywordTok{ts}\NormalTok{(AirPassengers) }\CommentTok{# convert to ts object}
\NormalTok{model =}\StringTok{ }\KeywordTok{ar}\NormalTok{(data, }\DataTypeTok{order=}\DecValTok{20}\NormalTok{)}
\NormalTok{forecast =}\StringTok{ }\KeywordTok{forecast}\NormalTok{(model, }\DataTypeTok{h=}\DecValTok{365}\NormalTok{)}
\KeywordTok{plot}\NormalTok{(forecast)}
\end{Highlighting}
\end{Shaded}

\includegraphics{day1-methods_files/figure-beamer/unnamed-chunk-11-1.pdf}

\end{block}

\begin{block}{Predict using auto.ARIMA}

\[AR + I + MA\]

or rather

\[AR(x) + I(x) + MA(x)\] where \(x\) is the `order' (plain english: how
often it is applied)

\end{block}

\begin{block}{What is \(I\)?}

\begin{Shaded}
\begin{Highlighting}[]
\NormalTok{data =}\StringTok{ }\KeywordTok{ts}\NormalTok{(AirPassengers) }\CommentTok{# convert to ts object}
\NormalTok{diff =}\StringTok{ }\KeywordTok{diff}\NormalTok{(data, }\DataTypeTok{differences =} \DecValTok{1}\NormalTok{) }\CommentTok{# create I of n-th order}
\KeywordTok{plot}\NormalTok{(diff)}
\end{Highlighting}
\end{Shaded}

\includegraphics{day1-methods_files/figure-beamer/unnamed-chunk-12-1.pdf}

\end{block}

\begin{block}{Predict using auto.ARIMA (4 years)}

\begin{Shaded}
\begin{Highlighting}[]
\NormalTok{data =}\StringTok{ }\KeywordTok{ts}\NormalTok{(AirPassengers)}
\NormalTok{model =}\StringTok{ }\KeywordTok{auto.arima}\NormalTok{(data)}
\NormalTok{forecast =}\StringTok{ }\KeywordTok{forecast}\NormalTok{(model, }\DataTypeTok{h=}\DecValTok{30}\NormalTok{)}
\KeywordTok{plot}\NormalTok{(forecast)}
\end{Highlighting}
\end{Shaded}

\includegraphics{day1-methods_files/figure-beamer/unnamed-chunk-13-1.pdf}

\end{block}

\begin{block}{Long-Term Forecast using auto.ARIMA}

\begin{Shaded}
\begin{Highlighting}[]
\NormalTok{data =}\StringTok{ }\KeywordTok{ts}\NormalTok{(AirPassengers)}
\NormalTok{model =}\StringTok{ }\KeywordTok{auto.arima}\NormalTok{(data)}
\NormalTok{forecast =}\StringTok{ }\KeywordTok{forecast}\NormalTok{(model, }\DataTypeTok{h=}\DecValTok{120}\NormalTok{)}
\KeywordTok{plot}\NormalTok{(forecast)}
\end{Highlighting}
\end{Shaded}

\includegraphics{day1-methods_files/figure-beamer/unnamed-chunk-14-1.pdf}

\end{block}

\end{frame}

\begin{frame}[fragile]{Practical Challenge}

\begin{block}{Predict \texttt{mdeaths}}

\begin{itemize}
\tightlist
\item
  Predict \texttt{mdeaths} with \texttt{auto.arima} for 3 years
\item
  \texttt{mdeaths} is split into months, hence one year equals
  \texttt{h=12}
\item
  The dataset:
\end{itemize}

\begin{Shaded}
\begin{Highlighting}[]
\NormalTok{data =}\StringTok{ }\KeywordTok{ts}\NormalTok{(mdeaths)}
\end{Highlighting}
\end{Shaded}

\end{block}

\begin{block}{Solution}

\begin{Shaded}
\begin{Highlighting}[]
\NormalTok{data =}\StringTok{ }\KeywordTok{ts}\NormalTok{(mdeaths)}
\NormalTok{model =}\StringTok{ }\KeywordTok{auto.arima}\NormalTok{(data)}
\NormalTok{forecast =}\StringTok{ }\KeywordTok{forecast}\NormalTok{(model, }\DataTypeTok{h=}\DecValTok{36}\NormalTok{)}
\KeywordTok{plot}\NormalTok{(forecast)}
\end{Highlighting}
\end{Shaded}

\includegraphics{day1-methods_files/figure-beamer/unnamed-chunk-16-1.pdf}

\end{block}

\begin{block}{What to do?}

\begin{itemize}
\tightlist
\item
  Change forecast method
\item
  Transform data
\item
  Tune \(AR(x) + I(x) + MA(x)\) orders manually
\end{itemize}

\end{block}

\end{frame}

\begin{frame}[fragile]{Recap}

\begin{block}{Recap}

\begin{itemize}
\tightlist
\item
  Forecasting can be as simple as taking the average
\item
  More complex forecasting follows the same approach: extrapolating from
  the past
\item
  The \texttt{forecast} package is the holy grail of forecasting
\item
  \texttt{auto.arima()} provides a way to do automated forecasting with
  classic methods
\end{itemize}

\end{block}

\end{frame}

\begin{frame}[fragile]{Cutting-edge forecasting with \texttt{prophet}}

\begin{block}{Traditional Methods}

\begin{itemize}
\tightlist
\item
  Not made for large-scale data
\item
  Need a lot of manual inspection and tuning
\item
  Sensitive to outliers and missing data
\item
  Requires an in-house specialist in \textbf{each} department
\item
  In short: costly to deploy \& to scale
\end{itemize}

\end{block}

\begin{block}{Modern Methods}

\begin{itemize}
\tightlist
\item
  Generalised Additive Models
\item
  Memoryless Neural Networks (NN)
\item
  Recurrent Neural Networks (RNN/LTSM)
\end{itemize}

\end{block}

\begin{block}{Modern Method: \texttt{prophet}}

\includegraphics{https://research.fb.com/wp-content/themes/fb-research/images/branding/FB_logo.svg}
\url{https://research.fb.com/prophet-forecasting-at-scale/}

\end{block}

\begin{block}{Modern Method: \texttt{prophet}}

\begin{Shaded}
\begin{Highlighting}[]
\NormalTok{m =}\StringTok{ }\KeywordTok{prophet}\NormalTok{(df)}
\NormalTok{future <-}\StringTok{ }\KeywordTok{make_future_dataframe}\NormalTok{(m, }\DataTypeTok{periods =} \DecValTok{12}\NormalTok{, }\DataTypeTok{freq =} \StringTok{'month'}\NormalTok{)}
\NormalTok{forecast <-}\StringTok{ }\KeywordTok{predict}\NormalTok{(m, future)}
\end{Highlighting}
\end{Shaded}

\end{block}

\begin{block}{Modern Method: \texttt{prophet}}

\begin{Shaded}
\begin{Highlighting}[]
\KeywordTok{plot}\NormalTok{(m, forecast)}
\end{Highlighting}
\end{Shaded}

\includegraphics{day1-methods_files/figure-beamer/unnamed-chunk-19-1.pdf}

\end{block}

\begin{block}{Modern Method: \texttt{prophet}}

\begin{Shaded}
\begin{Highlighting}[]
\KeywordTok{prophet_plot_components}\NormalTok{(m, forecast)}
\end{Highlighting}
\end{Shaded}

\includegraphics{day1-methods_files/figure-beamer/unnamed-chunk-20-1.pdf}

\end{block}

\begin{block}{How Does \texttt{prophet} Work?}

\begin{itemize}
\tightlist
\item
  \textbf{NOT} AutoRegressive (rather, finds a \emph{curve})
\item
  Robust against

  \begin{itemize}
  \tightlist
  \item
    missing data
  \item
    outliers
  \item
    significant trend changes
  \end{itemize}
\item
  Fully automatic
\item
  Fast at internet-scale data
\end{itemize}

\[Y = level + trend + seasonality + holidays + noise\]

\end{block}

\end{frame}

\begin{frame}[fragile]{Practical Challenge}

\begin{itemize}
\tightlist
\item
  Open \texttt{day1/data/prophet-AirPassengers-long.R}
\item
  Run the file

  \begin{itemize}
  \tightlist
  \item
    it will predict 10 years using

    \begin{itemize}
    \tightlist
    \item
      \texttt{prophet} and
    \item
      \texttt{auto.arima}
    \end{itemize}
  \end{itemize}
\item
  You will receive two plots (use the arrows to navigate between them)
\item
  Which forecast is more sensible?
\end{itemize}

\end{frame}

\begin{frame}[fragile]{Recap}

\begin{block}{Recap}

\begin{itemize}
\tightlist
\item
  Old methods rule the game
\item
  \ldots{}but modern methods make way for easier forecasting - they
  \emph{democratise} forecasting
\item
  For example, \texttt{prophet} offers a low-cost, high-scale platform
\item
  A cost-effcient way to experiment (and to forecast, too)
\item
  you can do \textbf{low-cost} experiments with \textbf{automated}
  methods, use e.g.

  \begin{itemize}
  \tightlist
  \item
    \texttt{prophet}
  \item
    \texttt{auto.arima()} from the \texttt{forecast} package
  \end{itemize}
\item
  Try both!
\end{itemize}

\end{block}

\end{frame}

\end{document}
